\documentclass[letterpaper,11pt,oneside,reqno]{amsart}
\usepackage[T2A]{fontenc}%
\usepackage[utf8]{inputenc}%
\usepackage[english]{babel}%
\usepackage{amsmath,amssymb,amsthm,amsfonts,upgreek}%
\usepackage{hyperref}%
\usepackage{enumerate}%
\usepackage{array}%
\usepackage{graphicx}
\usepackage[mathscr]{euscript}
\usepackage{color,tikz}
\usetikzlibrary{shapes}
\usepackage[DIV15]{typearea}
\usepackage[width=.9\textwidth]{caption}
\allowdisplaybreaks%
\numberwithin{equation}{section}%
\usepackage{listings}

%%%%%%%%%%%%%%%%%%%%%%%%%%%%%%%%%%%%%%%%%%%%%%%%%%%%%%%%%%%%
%%% Graphic primitives

%%%%%%%%%%%%%%%%%%%%%%%%%%%%%%%%%%%%%%%%%%%%%%%%%%%%%%%%%%%%
%draft-specific
\synctex=1%
\newcommand{\note}[1]{\textsc{\color{blue}(#1)}}
% \usepackage{refcheck}
%KILL TIKZ
	% \usepackage{environ}
	% \makeatletter
	% % \providecommand{\env@tikzpicture@save@env}{}
	% % \providecommand{\env@tikzpicture@process}{}
	% \RenewEnviron{tikzpicture}[1][]{notikz}
	% \makeatother
 
%%%%%%%%%%%%%%%%%%%%%%%%%%%%%%%%%%%%%%%%%%%%%%%%%%%%%%%%%%%%
\newtheorem{proposition}{Proposition}[section]
\newtheorem{lemma}[proposition]{Lemma}
\newtheorem{corollary}[proposition]{Corollary}
\newtheorem{theorem}[proposition]{Theorem}
%%%%%%%%%%%%%%%%%%%%%%%%%%%%%%%%%%%%%%%%%%%%%%%%%%%%%%%%%%%%
\theoremstyle{definition}
\newtheorem{definition}[proposition]{Definition}
\newtheorem{remark}[proposition]{Remark}
\newtheorem{example}[proposition]{Example}
\newtheorem{exercise}[proposition]{Exercise}
%%%%%%%%%%%%%%%%%%%%%%%%%%%%%%%%%%%%%%%%%%%%%%%%%%%%%%%%%%%%

\begin{document}

\title[Notes on random matrices]{Notes on random matrices}

\author[L. Petrov]{Leonid Petrov}
\address{L. Petrov, Department of Mathematics, University of Virginia, 
141 Cabell Drive, Kerchof Hall,
P.O. Box 400137,
Charlottesville, VA 22904, USA,
\newline{}and Institute for Information Transmission Problems, Bolshoy Karetny per. 19, Moscow, 127994, Russia}
\email{lenia.petrov@gmail.com}
\date{}
\maketitle

\begin{center}
	(notes by \note{students in MATH 8380 course})
\end{center}

\tableofcontents
\setcounter{tocdepth}{3}

\section*{\note{on the \TeX{} style}}

\lstset{basicstyle=\footnotesize\ttfamily,language=TeX} 

\textbf{1.} Please do not define and use any \lstinline{\newcommand} commands!

\textbf{2.} Try to be consistent --- use environments provided:
\begin{lstlisting}
	\begin{theorem}
		A theorem.		
	\end{theorem}
	\begin{proof}[Idea of proof]
		And here is an equation:
		\begin{align*}
			a^2+b^2=c^2.
		\end{align*}
		This concludes the proof.
	\end{proof}	
\end{lstlisting}
The above code produces:
\begin{theorem}
	A theorem.		
\end{theorem}
\begin{proof}[Idea of proof]
	And here is an equation:
	\begin{align*}
		a^2+b^2=c^2.
	\end{align*}
	This concludes the proof.
\end{proof}
Use \lstinline{align} or \lstinline{multline} for displayed equations.

\textbf{3.} Full list of theorem environments defined:
\begin{lstlisting}
	\newtheorem{proposition}{Proposition}[section]
	\newtheorem{lemma}[proposition]{Lemma}
	\newtheorem{corollary}[proposition]{Corollary}
	\newtheorem{theorem}[proposition]{Theorem}
	\newtheorem{definition}[proposition]{Definition}
	\newtheorem{remark}[proposition]{Remark}
	\newtheorem{example}[proposition]{Example}
	\newtheorem{exercise}[proposition]{Exercise}	
\end{lstlisting}

\textbf{4.} Use \lstinline{\note} command to insert notes: \note{this is a note}.

\section{Introduction} % (fold)
\label{sec:introduction}

% section introduction (end)







\bibliography{bib}
\bibliographystyle{amsalpha}

\end{document}