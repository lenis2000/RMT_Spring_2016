\documentclass[letterpaper,11pt,oneside,reqno]{amsart}
\usepackage[T2A]{fontenc}
\usepackage[utf8]{inputenc}
\usepackage[english]{babel}
\usepackage{amsmath,amssymb,amsthm,amsfonts,upgreek}
\usepackage{hyperref}
\usepackage{enumerate}
\usepackage{graphicx}
\usepackage[mathscr]{euscript}
\usepackage{color,tikz}
\usepackage[DIV15]{typearea}
\usepackage[width=.9\textwidth]{caption}
\allowdisplaybreaks
\numberwithin{equation}{section}
\usepackage{listings}

\synctex=1
\newcommand{\note}[1]{\textsc{\color{blue}(#1)}}
\newcounter{lecture}
\newcommand{\lect}[1]{\medskip\addtocounter{lecture}{1}\noindent{\Large\textbf{\color{red}Lecture \#\arabic{lecture} on #1 \hrulefill}}\medskip}
\usepackage{refcheck}
 
%%%%%%%%%%%%%%%%%%%%%%%%%%%%%%%%%%%%%%%%%%%%%%%%%%%%%%%%%%%%
\newtheorem{proposition}{Proposition}[section]
\newtheorem{lemma}[proposition]{Lemma}
\newtheorem{corollary}[proposition]{Corollary}
\newtheorem{theorem}[proposition]{Theorem}
\theoremstyle{definition}
\newtheorem{definition}[proposition]{Definition}
\newtheorem{remark}[proposition]{Remark}
\newtheorem{example}[proposition]{Example}
\newtheorem{exercise}[proposition]{Exercise}
%%%%%%%%%%%%%%%%%%%%%%%%%%%%%%%%%%%%%%%%%%%%%%%%%%%%%%%%%%%%

\begin{document}

\title[Notes on random matrices]{Notes on random matrices}

\author[L. Petrov]{Leonid Petrov}
\date{\today}
\maketitle

\begin{center}
	(notes by Bryce Terwilliger; \ldots)
\end{center}

\begin{abstract}
	These are notes for the MATH 8380 ``Random Matrices'' course at the University of Virginia in Spring 2016.
	The notes are constantly updated, and the latest version can be found at the \texttt{git}
	repository
	\url{https://github.com/lenis2000/RMT_Spring_2016}
\end{abstract}

\note{before \TeX{}ing, please familiarize yourself with style suggestions at 
\url{https://github.com/lenis2000/RMT_Spring_2016/blob/master/TeXing.md}}

\tableofcontents
\setcounter{tocdepth}{3}

\lect{1/20/2016}

\section{Introduction} % (fold)
\label{sec:introduction}

\subsection{Nonrandom and random matrices and their eigenvalues} % (fold)
\label{sub:object_of_study}

The study of random matrices as a field is a patchwork of many fields.  The
main object we study is a probability distribution on a certain subset of the
set of matrices $\mathrm{Mat}(N\times N,\ \mathbb R \text{ or } \mathbb C)$,
thus giving us a random matrix $A$.

\begin{definition}
An {\it eigenvalue} $\lambda$ of the matrix $A$ is a root of the polynomial
$f(\lambda)=\text{det}(A-\lambda I)$.  Equivalently, $\lambda$ is an
eigenvalue if $A$ if the matrix $A-\lambda I$ is not invertible. This second
way of defining eigenvalues in fact works even when $A$ is not a finite
size matrix, but an operator in some infinite-dimensional space.
\end{definition}

We will largely be only concerned with real eigenvalues.  That is the
eigenvalues of a real symmetric matrix over $\mathbb R$ or Hermitian over
$\mathbb C$ that is where $A^*=A$.

\begin{remark}
	The case when eigenvalues can be complex is also studied in
	the theory of random matrices, sometimes under the keyword \emph{complex
	random matrices}. See, for example, \cite{gotze2010circular} for a law of 
	large numbers for complex eigenvalues.
\end{remark}

\begin{proposition}
Every eigenvalue of a Hermitian matrix is real.
\end{proposition}
\begin{proof}
Let $A$ be a Hermitian matrix so that $A^*=A$ (here and everywhere below
$A*$ means $\overline{A^{\text{T}}}$, i.e., transposition and complex conjugation).
Let $\lambda$ be an eigenvalue of $A$.  Let $v$ be a non-zero vector in the null
space of $A-\lambda I$.  Let $a=\overline{ v^{\text{T}}}v=|v|^2$, so
that $a$ is a positive real number.  Let $b=\overline{ v^{\text{T}}}A
v$.  Then $\bar b=\overline{ b^{\text{T}}}=\overline{\overline{
v^{\text{T}}}Av}^{\text{T}}=\overline{
v^{\text{T}}}\bar A^{\text{T}} v=\overline{
v^{\text{T}}}Av=b$, so $b$ is real.  Since $b=\lambda a$, $\lambda$
must be real. 
\end{proof}

Let $\mathcal H_N$ be the set of $N\times N$ Hermitian matrices.  For each
$N$, let $\mu_N$ be a probability measure on $\mathcal H_N$ (it can be
supported not by  the whole $\mathcal H_N$, but by a subset of it, too).  Then
for each matrix $A\in \mathcal H_N$ we may order the real eigenvalues
$\lambda_1\geq \cdots \geq \lambda_N$ of $A$.

A collection of probability measures $\mu_N$ on $\mathcal H_N$ for each
$N\ge1$ is said to be a \emph{random matrix ensemble}. For such an ensemble,
the eigenvalues $\lambda_1^{(N)}\geq \cdots \geq \lambda_N^{(N)}$ of matrices
$N$ form random point configurations on $\mathbb{R}$ with growing numbers of points.
Our main goal is to study the asymptotic properties of these collections of points on $\mathbb{R}$,
as $N\to\infty$.

% subsection object_of_study (end)

% section introduction (end)






\bibliography{bib}
\bibliographystyle{amsalpha}

\end{document}